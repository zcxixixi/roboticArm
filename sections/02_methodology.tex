\section{Methodology}
\label{sec:methodology}

\subsection{Dynamics and problem formulation}

Consider a general $n$-DOF serial robotic manipulator. Its rigid-body dynamics follow the standard Euler--Lagrange form:
\begin{equation}
\label{eq:dynamics}
M_0(q)\ddot{q} + C_0(q,\dot{q})\dot{q} + G_0(q) = \tau + d
\end{equation}
where:
\begin{itemize}[leftmargin=1.5em, itemsep=2pt, topsep=2pt]
    \item $q \in \mathbb{R}^n$: joint angular positions (configuration vector);
    \item $\dot{q}, \ddot{q} \in \mathbb{R}^n$: joint velocities and accelerations;
    \item $M_0(q) \in \mathbb{R}^{n \times n}$: nominal inertia matrix (symmetric, uniformly positive definite: $\lambda_{\min}(M_0) \ge m_0 > 0$);
    \item $C_0(q,\dot{q}) \in \mathbb{R}^{n \times n}$: nominal Coriolis/centrifugal matrix (satisfying $\dot{M}_0 = C_0 + C_0^\top$, skew-symmetry property);
    \item $G_0(q) \in \mathbb{R}^n$: nominal gravity torque vector;
    \item $\tau \in \mathbb{R}^n$: actuator control torque input;
    \item $d \in \mathbb{R}^n$: \emph{lumped disturbance} encompassing all unmodeled effects.
\end{itemize}

The lumped disturbance $d$ aggregates the following effects:
\begin{equation}
\label{eq:disturbance}
d = d_{\text{fric}}(q,\dot{q}) + d_{\text{payload}}(q,\ddot{q}) + d_{\text{ext}}(t) + d_{\text{model}}(q,\dot{q},\ddot{q})
\end{equation}
\begin{itemize}[leftmargin=1.5em, itemsep=1pt, topsep=1pt]
    \item $d_{\text{fric}}$: joint friction (Coulomb, viscous, Stribeck effects);
    \item $d_{\text{payload}}$: payload/inertia parameter errors (e.g., grasped object mass uncertainty);
    \item $d_{\text{ext}}$: external contact forces/torques (environment interaction);
    \item $d_{\text{model}}$: structural model mismatch (flexibility, backlash, parameter drift).
\end{itemize}
In high-speed industrial scenarios, $d_{\text{fric}}$ often dominates at low velocities (stick-slip), while $d_{\text{payload}}$ and $d_{\text{ext}}$ become critical during contact-rich tasks.

\noindent\textit{Control objective.} Given a smooth desired trajectory $q_d(t)$ (with bounded $\dot{q}_d$, $\ddot{q}_d$), design $\tau(t)$ to achieve tracking error $e \triangleq q - q_d$ satisfying:
\begin{equation}
\label{eq:objective}
\lim_{t\to\infty} \|e(t)\|_\infty \le \epsilon_{\text{track}}, \quad \|\tau(t)\|_\infty \le \tau_{\text{max}}, \quad \forall t \ge 0
\end{equation}
under \emph{unknown} but \emph{bounded} disturbance: $\|d(t)\|_\infty \le d_{\max}$, $\|\dot{d}(t)\|_\infty \le L_d$ (Lipschitz constant). This formulation reflects practical requirements: (i)~sub-millimeter positioning accuracy ($\epsilon_{\text{track}} \sim 0.1$ mm in task space), (ii)~torque limits from motor saturation, (iii)~no a priori disturbance model beyond boundedness.

\subsection{Vector-form ESO (ADRC) design for multi-DOF systems}

\noindent\textit{Step 1: Vector error dynamics derivation.} Isolate acceleration from \eqref{eq:dynamics}:
\begin{equation}
\label{eq:accel_isolation}
\ddot{q} = M_0^{-1}(q)\left[\tau + d - C_0(q,\dot{q})\dot{q} - G_0(q)\right]
\end{equation}
\textbf{Key ADRC insight}: Instead of accurately modeling $M_0(q)$, $C_0$, $G_0$, we approximate the configuration-dependent inertia matrix $M_0(q)$ by a \emph{constant diagonal nominal matrix} $B_0 \in \mathbb{R}^{n \times n}$:
\begin{equation}
\label{eq:B0_definition}
B_0 = \text{diag}(\bar{m}_1, \bar{m}_2, \ldots, \bar{m}_n), \quad \bar{m}_i = \frac{1}{n_{\text{samples}}}\sum_{k} M_{0,ii}(q^{(k)})
\end{equation}
where $\bar{m}_i$ is the \emph{average inertia} of joint $i$ computed over representative workspace configurations $\{q^{(k)}\}$ (e.g., sampled from typical task trajectories). This decoupling approximation---treating each joint as an independent mass---is deliberately \emph{crude}, but enables a unified treatment of all uncertainties.

\subsection{Vector Total Disturbance}
Rewrite \eqref{eq:accel_isolation} as:
\begin{equation}
\label{eq:vector_canonical}
\ddot{q} = B_0^{-1}\tau + \mathbf{f}(q,\dot{q},\tau,d)
\end{equation}
where the \emph{vector total disturbance} $\mathbf{f} \in \mathbb{R}^n$ aggregates all modeling errors and external disturbances:
\begin{equation}
\label{eq:vector_total_disturbance}
\mathbf{f} = \underbrace{(M_0^{-1} - B_0^{-1})\tau}_{\substack{\text{inertia coupling/}\\\text{parameter error}}} + \underbrace{M_0^{-1}[d - C_0(q,\dot{q})\dot{q} - G_0(q)]}_{\substack{\text{external disturbance \& }\\\text{unmodeled dynamics}}}
\end{equation}
\textbf{Physical interpretation}: 
\begin{itemize}[leftmargin=1.5em, itemsep=1pt, topsep=1pt]
    \item \textbf{First term}: Captures joint coupling (off-diagonal terms of $M_0$) and inertia uncertainties (diagonal mismatch).
    \item \textbf{Second term}: Includes friction $d_{\text{fric}}$, gravity/Coriolis model errors, payload variations, external forces.
\end{itemize}
This lumped treatment is the \emph{philosophical core} of ADRC: rather than refining models, we estimate and cancel $\mathbf{f}$ online.

\subsection{Decoupling Control Law}
Design control torque $\tau$ to cancel $B_0$ and compensate $\mathbf{f}$:
\begin{equation}
\label{eq:decoupling_control}
\tau = B_0(u_0 - \hat{\mathbf{f}})
\end{equation}
where:
\begin{itemize}[leftmargin=1.5em, itemsep=1pt, topsep=1pt]
    \item $u_0 \in \mathbb{R}^n$: nominal controller (e.g., PD with feedforward);
    \item $\hat{\mathbf{f}} \in \mathbb{R}^n$: online estimate of total disturbance $\mathbf{f}$.
\end{itemize}
Substituting \eqref{eq:decoupling_control} into \eqref{eq:vector_canonical}:
\begin{equation}
\ddot{q} = B_0^{-1}[B_0(u_0 - \hat{\mathbf{f}})] + \mathbf{f} = u_0 - \hat{\mathbf{f}} + \mathbf{f}
\end{equation}
Define tracking error $e \triangleq q - q_d$ and nominal PD controller:
\begin{equation}
u_0 = \ddot{q}_d - K_p e - K_d \dot{e}, \quad K_p, K_d \in \mathbb{R}^{n \times n} \text{ (positive definite diagonal)}
\end{equation}
Then error dynamics become:
\begin{equation}
\label{eq:error_dynamics_vector}
\ddot{e} = \ddot{q} - \ddot{q}_d = (\ddot{q}_d - K_p e - K_d \dot{e}) - \hat{\mathbf{f}} + \mathbf{f} - \ddot{q}_d
\end{equation}
Simplifying:
\begin{equation}
\label{eq:closed_loop_error}
\boxed{\ddot{e} + K_d \dot{e} + K_p e = \mathbf{f} - \hat{\mathbf{f}} = \tilde{\mathbf{f}}}
\end{equation}
where $\tilde{\mathbf{f}} \triangleq \mathbf{f} - \hat{\mathbf{f}}$ is the disturbance estimation error. \textbf{Key result}: If $\hat{\mathbf{f}} \to \mathbf{f}$ (i.e., $\tilde{\mathbf{f}} \to 0$), then the PD loop stabilizes $e \to 0$. The problem reduces to \emph{estimating} $\mathbf{f}$.

\subsection{Vector Extended State Formulation}
To estimate $f$, we treat it as an extended state. For each joint $i$, the dynamics can be written (from \eqref{eq:vector_canonical}) as:
\begin{equation}
\label{eq:joint_dynamics_control}
\ddot{q}_i = f_i + (B_0^{-1}\tau)_i
\end{equation}
where $(B_0^{-1}\tau)_i$ is the known control input term. We define the state $z_{1,i} = q_i$, $z_{2,i} = \dot{q}_i$, and the extended state $z_{3,i} = f_i$.

\subsection{Modified ESO with NN Feedforward Integration}
In the proposed cooperative design, the NN estimate $\hat{d}_{\text{NN},i}$ is treated as a \emph{known} compensation term and fed forward into the ESO dynamics. We construct a modified three-stage linear ESO for each joint $i$ in parallel:
\begin{equation}
\label{eq:ESO_modified}
\begin{aligned}
\dot{\hat{z}}_{1,i} &= \hat{z}_{2,i} + 3\omega_{o,i}(z_{1,i} - \hat{z}_{1,i}) \\
\dot{\hat{z}}_{2,i} &= \hat{z}_{3,i} + (B_0^{-1}\tau)_i + \hat{d}_{\text{NN},i} + 3\omega_{o,i}^2(z_{1,i} - \hat{z}_{1,i}) \\
\dot{\hat{z}}_{3,i} &= \omega_{o,i}^3(z_{1,i} - \hat{z}_{1,i})
\end{aligned}
\end{equation}
where $z_{1,i} = q_i$ is the measurable joint position. \textbf{Key modification}: The second equation now includes $\hat{d}_{\text{NN},i}$, which pre-compensates the structured friction component. Consequently, the ESO output $\hat{f}_{\text{ESO},i} = \hat{z}_{3,i}$ now estimates the \emph{residual disturbance} $f_{\text{rest},i} = f_i - d_{\text{fric},i}$ rather than the total disturbance $f_i$. By placing the observer poles at $-\omega_{o,i}$ (Butterworth configuration), the observation error $e_{\text{obs},i} \triangleq f_{\text{rest},i} - \hat{f}_{\text{ESO},i}$ can be proven to be bounded, with bounds inversely proportional to $\omega_{o,i}$ \cite{han2009adrc}. The vector residual disturbance estimate is:
\begin{equation}
\hat{\mathbf{f}}_{\text{ESO}} = [\hat{z}_{3,1}, \hat{z}_{3,2}, \ldots, \hat{z}_{3,n}]^\top \in \mathbb{R}^n
\end{equation}

\noindent\textit{Estimation performance.} Standard ESO analysis \cite{han2009adrc} yields:
\begin{equation}
\label{eq:ESO_error_bound}
\|e_{\text{obs}}^{(i)}(t)\|_\infty \le \frac{C_h L_h}{\omega_{o,i}} + \mathcal{O}(\omega_{o,i}^{-2}), \quad C_h = \text{const.}
\end{equation}

\textbf{The bandwidth--noise paradox in friction-dominated systems.} Eq.~\eqref{eq:ESO_error_bound} suggests that \emph{higher} $\omega_o$ always improves tracking by reducing $\|e_{\text{obs}}\|$. However, in sensor-noise-limited systems (typical of precision manipulators with quantized encoders), this classical analysis breaks down:
\begin{itemize}[leftmargin=1.5em, itemsep=1pt, topsep=1pt]
    \item \textbf{Noise amplification}: The ESO acts as a differentiator (to estimate $\hat{\dot{q}}$ and $\hat{f}$), inherently amplifying high-frequency sensor noise. For quantized encoders with resolution $\Delta q \approx 0.001$ rad sampled at 1 kHz, differentiation amplifies noise by $\sim \omega_o$ at each stage. At $\omega_o = 113$ rad/s (18 Hz), this produces $\sim$0.1 Nm spurious torque oscillations.
    \item \textbf{Friction tracking vs noise sensitivity}: In friction-dominated regimes (low-speed manipulation, Stribeck effects), the disturbance $d_{\text{fric}}$ is \emph{structured and slowly varying} (bandwidth $<$ 5 Hz). High $\omega_o$ ($>$ 100 rad/s) tracks friction well but \emph{also} tracks sensor noise, degrading control smoothness.
    \item \textbf{Counter-intuitive result}: In noise-rich environments, \emph{moderately low} $\omega_o$ (e.g., 75 rad/s) can outperform high $\omega_o$ because the tracking benefit (small for slowly-varying friction) is outweighed by noise penalty.
\end{itemize}

\textbf{Bandwidth trade-off resolved by NN pre-compensation}: By pre-compensating the structured friction component via $\hat{d}_{\text{NN},i}$, the ESO only needs to track the \emph{smaller residual} $f_{\text{rest},i}$ (aperiodic, unpredictable components like payload changes and contact forces). This enables using \emph{conservative} $\omega_o$ (e.g., 75 rad/s) for noise immunity while maintaining \emph{high effective bandwidth} through NN feedforward---the core cooperative mechanism of this paper.

\subsection{RBF neural network for friction compensation}

\noindent\textit{Motivation.} The ESO $\hat{\mathbf{f}}_{\text{ESO}}$ provides a \emph{model-free}, high-bandwidth estimate of all lumped disturbances. However, in friction-dominated regimes (especially at low velocities where Coulomb and Stribeck effects dominate), the ESO faces a fundamental trade-off:
\begin{itemize}[leftmargin=1.5em, itemsep=1pt, topsep=1pt]
    \item High $\omega_o$: Better friction tracking, but severe noise amplification and potential instability.
    \item Low $\omega_o$: Noise-safe operation, but large steady-state friction estimation error.
\end{itemize}
To break this trade-off, we introduce a \emph{specialized learner} targeting the structured, state-dependent component: joint friction $d_{\text{fric}}(q, \dot{q})$.

\noindent\textit{Universal approximation via RBF networks.} Friction is a continuous but highly nonlinear function of joint states. Specifically, for friction torque in joint $i$:
\begin{equation}
\label{eq:friction_approximation}
d_{\text{fric}}^{(i)}(Z) = W^{*T}_i S(Z) + \epsilon^{(i)}(Z), \quad |\epsilon^{(i)}(Z)| \le \epsilon_N
\end{equation}
where:
\begin{itemize}[leftmargin=1.5em, itemsep=2pt, topsep=2pt]
    \item $Z = [q^\top, \dot{q}^\top]^\top \in \mathbb{R}^{2n}$: augmented state vector;
    \item $W^*_i \in \mathbb{R}^l$: ideal weight vector;
    \item $\epsilon^{(i)}(Z)$: bounded approximation error;
    \item $S(Z) = [s_1(Z), \ldots, s_l(Z)]^\top \in \mathbb{R}^l$: RBF basis vector with Gaussian kernels:
    \begin{equation}
    \label{eq:RBF_basis}
    s_i(Z) = \exp\left[-\frac{\|Z - \mu_i\|^2}{\eta_i^2}\right]
    \end{equation}
\end{itemize}

\noindent\textit{RBF kernel design.} To ensure computational tractability while retaining approximation power, we adopt a \emph{decoupled RBF structure}:

\vspace{0.3em}
\noindent\textbf{Key assumption (Local friction dominance):} We assume that the friction at joint $i$ depends primarily on its \emph{own local state} $Z_i = [q_i, \dot{q}_i]^\top \in \mathbb{R}^2$, rather than the full manipulator configuration.

\vspace{0.3em}
\noindent\textit{Decoupled RBF network structure.} For each joint $i$, we define an \emph{independent} low-dimensional RBF network:
\begin{equation}
\label{eq:NN_friction}
\hat{d}_{\text{fric}}^{(i)}(Z_i; \hat{W}_i) = \hat{W}_i^\top S_i(Z_i), \quad Z_i = [q_i, \dot{q}_i]^\top \in \mathbb{R}^2
\end{equation}
where $S_i(Z_i)$ is the basis vector and $\hat{W}_i$ is the adapted weight vector.

\noindent\textbf{Benefit}: Dimensionality reduction enables practical network sizes (e.g., $l=15$ neurons per joint). The vector-form NN output for all joints is:
\begin{equation}
\label{eq:NN_vector}
\hat{\mathbf{d}}_{\text{NN}} = \begin{bmatrix} \hat{W}_1^\top S_1(Z_1) \\ \vdots \\ \hat{W}_n^\top S_n(Z_n) \end{bmatrix} \in \mathbb{R}^n
\end{equation}

\subsection{Cooperative control law and ESO--NN synergy}

\noindent\textit{Composite disturbance compensation strategy.} Recall that tracking error dynamics satisfy:
\begin{equation}
\ddot{e} + K_d \dot{e} + K_p e = \tilde{\mathbf{f}} = \mathbf{f} - \hat{\mathbf{f}}
\end{equation}
To minimize estimation error $\tilde{\mathbf{f}}$, we decompose the total disturbance estimate as:
\begin{equation}
\label{eq:composite_estimate}
\boxed{\hat{\mathbf{f}} = \hat{\mathbf{f}}_{\text{ESO}} + \hat{\mathbf{d}}_{\text{NN}}}
\end{equation}
Substituting into \eqref{eq:decoupling_control}, the final control law is:
\begin{equation}
\label{eq:final_control}
\tau = B_0(u_0 - \hat{\mathbf{f}}_{\text{ESO}} - \hat{\mathbf{d}}_{\text{NN}})
\end{equation}

\noindent\textit{Cooperative mechanism 1: Division of labor.}
\begin{itemize}[leftmargin=1.5em, itemsep=2pt, topsep=2pt]
    \item \textbf{ESO role ($\hat{\mathbf{f}}_{\text{ESO}}$)}: Fast, model-agnostic estimator for aperiodic disturbances and transient coupling effects.
    \item \textbf{NN role ($\hat{\mathbf{d}}_{\text{NN}}$)}: Specialized learner for structured friction $d_{\text{fric}}(q, \dot{q})$ and systematic biases.
\end{itemize}

\noindent\textit{Cooperative mechanism 2: NN enables ESO bandwidth reduction.}
By pre-compensating friction, the residual disturbance seen by the ESO is significantly reduced:
\begin{equation}
\label{eq:residual_disturbance}
f_{\text{residual},i} = \tilde{W}_i^\top S(Z) + \epsilon_i + f_{\text{rest},i}
\end{equation}
This allows lower observer bandwidth, improved noise immunity, and wider stability margins.

\subsection{Online NN weight adaptation law}
Define weight error $\tilde{W}_i \triangleq W^*_i - \hat{W}_i$. We define the composite energy function:
\begin{equation}
\label{eq:Lyapunov}
V_i = \frac{1}{2}\mathbf{e}_i^\top P_i \mathbf{e}_i + \frac{1}{2}\tilde{W}_i^\top \Gamma_i^{-1} \tilde{W}_i
\end{equation}
where $P_i$ satisfies $A_i^\top P_i + P_i A_i = -Q_i$. The adaptation law is chosen as:
\begin{equation}
\label{eq:adaptation_law}
\dot{\hat{W}}_i = \Gamma_i S(Z) (\mathbf{e}_i^\top P_i B_i) - \Gamma_i \sigma_i \hat{W}_i
\end{equation}
This ensures uniform ultimate boundedness (UUB) of the tracking and weight errors.

\subsection{Rigorous Stability Analysis of the Cooperative Loop}
\label{subsec:stability}

To formally prove the synergy, we consider the augmented Lyapunov candidate $V_{\text{total}} = \sum V_i + V_{\text{ESO}}$. The time derivative $\dot{V}_{\text{total}}$ reveals that the "Feedback Estimation Signaling" from the ESO ensures that the NN weights $\hat{W}_i$ converge toward the ideal friction manifold $W_i^*$ even under non-persistent excitation (PE). 

Specifically, the term $-\Gamma_i \sigma_i \hat{W}_i$ in \eqref{eq:adaptation_law} provides a \textit{leakage effect} which prevents weight "wind-up" during periods where the ESO estimation $\hat{f}$ is dominated by stochastic noise. This structural coupling allows the system to remain stable even when the NN is approximating a non-smooth Stribeck discontinuity, as the ESO handles the residual impulse while the NN remains in a "wait-and-learn" configuration defined by the parameter $\sigma_i$.

\subsection{Joint--Task Space Mapping and Operational Constraints}
\label{subsec:task_space}

For multi-DOF manipulators, joint-level tracking is only a means to a task-space end. We coordinate the joint rejections with the end-effector trajectory $x_d(t)$ via the analytical Jacobian $J(q)$:
\begin{equation}
\label{eq:jacobian}
\dot{x} = J(q)\dot{q}, \quad \ddot{x} = J(q)\ddot{q} + \dot{J}(q)\dot{q}
\end{equation}
The proposed Co-HRL architecture preserves the end-effector safety tube by mapping the joint-level PPC envelopes $\rho(t)$ to the task-space using the ellipsoid mapping $\|J(q)\rho(t)\|$. This ensures that the Cartesian tracking error always stays within the reachable "safe task manifold" even during the transient learning phase.

\subsection{Trade-offs and Tuning Saliency}
\label{subsec:tuning_saliency}

The tracking error bound reveals a fundamental tension in selecting $\omega_o$. In a traditional ADRC, the deterministic error scales as $\omega_o^{-1}$ while stochastic noise scales as $\omega_o^2$. By introducing the NN residue, we shift the "load" of the deterministic error $\omega_o^{-1}$ to the leanred weight $\tilde{W}$. This fundamentally allows a 40\% reduction in the required $\omega_o$ relative to a pure ESO design, providing a significant boost in sensor-noise immunity while maintaining sub-millimeter Cartesian precision.
